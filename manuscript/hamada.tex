\documentclass[./main]{subfiles} % これを最初に書く

% ここにnewtheorem,newenvironment,defなどを書く
\theoremstyle{definition}
\newtheorem{hamadadefi}{定義}[section]
\newtheorem{hamadaex}[hamadadefi]{例}
\newtheorem{hamadaqst}[hamadadefi]{問題}
\newtheorem{hamadaprop}[hamadadefi]{命題}
\newtheorem{hamadathm}[hamadadefi]{定理}
\newtheorem{hamadalem}[hamadadefi]{補題}
\renewcommand\proofname{\bf 証明}

\begin{document} %ここから文章を始める

\Chapter{無限の大きさ比べ(濱田)}
% 章だては
% 番号付けをしない場合はSection,Subsection,Subsubsectionで行う(大文字に注意!)
% 番号付けをする場合はsection,subsection,subsubsectionで行う(小文字に注意!)
% 和文の句読点は全角のカンマとピリオドを使ってください
\Section{はじめに}
この文章は,理学部数学科4年(2018年度)の濱田が,
五月祭学術企画「ますらぼ」のために作成したものです.
ますらぼでのミニ講演は10-15分程度であり,短い時間の中で
数学の魅力を伝えることはできても,数学をきちんと語るには
それなりの時間が必要です.そこで,講演では概要だけわかりやすく
説明して,詳しい中身に興味が湧いた方にはこの文章を読んでもらおう,
ということにしました.この文章で講演内容をどれだけ補えているかは
わかりませんが,できるだけわかりやすく書いたつもりです.
お暇なときにゆっくりお読みいただければ幸いです.
なお,本文中に高校数学で学ぶ記号が説明なしに登場します.
意味が分からなければ,調べていただくか,読み飛ばしていただいても差し支えありません.

\section{イントロダクション}
「無限」という言葉は,おそらく多くの方が耳にしたことがあると思います.
しかし,無限とは何か,ということをきちんと考えてみたことはあるでしょうか.
明鏡国語辞典によれば,無限とは「限度がないこと.果てがないこと.」
という意味だそうです.何も言っていませんね.
このように,無限とは私たちが思っている以上に難しい概念であり,
実はそれは数学の世界でも変わりません.
この文章では,無限の大きさ比べをテーマに,
数学ではどのように「無限」を扱っているのかをほんの少しでもわかってもらうことを
目指します.

\section{大きさ比べの準備}
この節では,無限の大きさを比べるために,集合と写像に関するいろいろな言葉を定義します
\footnote{数学の世界で無限と言うと,解析学に現れる極限といった「無限」もありますが,
ここでは集合の「無限」について考えることにします.}.
現代数学では,まず言葉や概念を準備して,それらの満たす性質を確認し,
定理を証明する,という方法をとるのが一般的です.
ここでも,その習慣に倣って進めていくことにしましょう.
ただし,必ずしも数学を専門としない方にとっては,
それだとわかりづらいかもしれないので,
できるだけ例や問題も絡めていきたいと思います.

\subsection{集合}
何かある性質を満たす「もの」全部を集めたものを集合と呼びます.
ただし,「ある性質」は一つの「もの」に対して
イエスかノーで判定できるものでなければなりません.
集合は以下のように記述します:
\[
\{x\mid xが満たす条件\},\ \{x,y,z\},\ \{a,b,c,\cdots\}
\]

\begin{hamadaex}
\label{shugo}
$P=\{x\mid xは図鑑番号1から151までの炎タイプのポケモン\}$は集合である.
一方,$Q=\{x\mid xはみんなが好きなポケモン\}$は集合とは言えない.
数学的な例を挙げると
\begin{itemize}
\item $\mathbb{N}=\{x\mid xは自然数\}=\{0,1,2,\cdots\}$
\footnote{高校までは「自然数は$1,2,3,\cdots$」であることになっていますが,
大学に来ると「自然数は$1,2,3,\cdots$」という流儀と
「自然数は$0,1,2,\cdots$」という流儀が存在します.
この文章では後者の0を含む流儀を採用することにします.
この事実からわかるように,自然数に0を入れるか否かという問題は,
数学的に自然数の本質を突くものではありません.}
\item $\mathbb{Z}=\{x\mid xは整数\}=\{0,\pm1,\pm2,\cdots\}$
\item $\mathbb{Q}=\{x\mid xは有理数\}$
\item $\mathbb{R}=\{x\mid xは実数\}$
\item $\mathbb{C}=\{x\mid xは複素数\}$
\end{itemize}
は集合である
\footnote{$\mathbb{N}$,$\mathbb{Z}$,$\mathbb{Q}$,$\mathbb{R}$,$\mathbb{C}$は黒板太字と呼ばれ,
大学数学でよく使われます.大変便利な記号なので,この文章でも使うことにします.}
が,$X=\{p\mid pは素数っぽい数\}$は集合とは言えない.
\end{hamadaex}

\begin{hamadaqst}
\label{shugoq}
例\ref{shugo}に倣って,図形に関するもので「集合と言えるもの」・「集合と言えないもの」を挙げよ.
\end{hamadaqst}

$A$を集合とするとき,$A$を構成する一つ一つの「もの」を,
集合$A$の元(あるいは要素)と呼びます.$a$が集合$A$の元であるとき,
$a$は$A$に属すると言い,このことを$a\in A$と書きます.
逆に,$a$が$A$に属さないことを$a\notin A$と書きます.
例えば,例\ref{shugo}に戻ると,
$リザードン\in P$,$ピカチュウ\notin P$,$2018\in\mathbb{N}$,$-273\notin\mathbb{N}$
となります.

また,2つの集合$A,B$の元が全く同じである,すなわち
$A$の元がすべて$B$に属し,かつ$B$の元がすべて$A$に属する
とき,$A$と$B$は等しいと言い,$A=B$と書きます.

\begin{hamadaex}
$\{x\in\mathbb{R}\mid x^2=1\}=\{\pm1\}$である.
ただし,$\{x\in\mathbb{R}\mid x^2=1\}$は
「実数$x$ (すなわち$x\in\mathbb{R}$)であって
$x^2=1$を満たすもの全部の集合」を表している.
\end{hamadaex}

次に定義する言葉は,集合と集合の間の包含関係を表す概念です.

\begin{hamadadefi}[部分集合]
集合$A,B$が「$x\in A$ならば$x\in B$」を満たすとき,
$A$は$B$の部分集合であると言い,このことを$A\subset B$と書く.
特に$A\subset B$かつ$A\neq B$が成り立つとき,
$A$は$B$の真部分集合であると言い,このことを$A\subsetneq B$と書く.
\end{hamadadefi}

\begin{hamadaex}
$A=\{ミズゴロウ, ヌマクロー, ラグラージ\}$,
$B=\{ミズゴロウ, ラグラージ\}$,$C=\{ヌマクロー\}$とすると,
$B\subset A$,$C\subsetneq A$などが成り立つが,
$C\subset B$は成り立たない.
\end{hamadaex}

部分集合の定義から「$A\subset B$かつ$B\subset A$ならば$A=B$」が成り立つことがわかります.
したがって,$A=B$を証明するには$A\subset B$かつ$B\subset A$であることを示せば
十分です.

ここで,大きさ比べの対象となる「無限個の元を持つ集合」を定義しておきましょう.

\begin{hamadadefi}[有限集合・無限集合]
集合$A$の元の個数が有限であるとき,$A$を有限集合と呼ぶ.
有限集合でない集合を無限集合と呼ぶ.
\end{hamadadefi}

無限であるということを「有限でないこと」と定義しました.
多くの数学の本もこのように定義していると思います.
{\footnotesize 結局何も言っていません.ずるいですね.}

\subsection{写像}
さて,集合は用意したものの,集合どうしを比べる「ものさし」が無いと,
大きさの比べようがありません.そこで,集合と集合の間の関係を定めるための
概念として「写像」というものを定義しましょう.

\begin{hamadadefi}[写像]
\label{map}
$A,B$を集合とする.
任意の$a\in A$に対し,ただ一つの元$b\in B$を対応させるものを,
$A$から$B$への写像と呼ぶ.
$f$が$A$から$B$への写像であることを$f\colon A\to B$と書き,
このとき$A$を定義域,$B$を値域と呼ぶ.
また,$f$によって$a\in A$が対応する$b\in B$を$f(a)$と書く.
\end{hamadadefi}

\begin{hamadaex}
\label{mapex}
$\mathbb{R}$から$\mathbb{R}$への写像$f$を次のように定める:
\[
f\colon\mathbb{R}\to\mathbb{R},\ x\mapsto 2x+1
\]
矢印$\mapsto$は,$x\in\mathbb{R}$に対し$2x+1\in\mathbb{R}$を対応させる,
という意味である.つまり,写像$f\colon\mathbb{R}\to\mathbb{R}$は一次関数に他ならない.
関数とは,一般に$\mathbb{R}$や$\mathbb{C}$を値域に持つ写像のことを言う.
一方,$x>0$に対し$g(x)=(xの平方根)$とすると,$g$は写像でない.
\end{hamadaex}

\begin{hamadaqst}
\label{mapq}
集合$X=\{0,1,2,3\}$に対し,
$\mathbb{N}$から$X$への写像を一つ挙げよ.
また,例\ref{mapex}に倣って,写像でないものも一つ挙げよ.
\end{hamadaqst}

次の定義は何だか回りくどいですが,要するに
「$f$が$A$の元と$B$の元を1対1に結び付ける」ということです.
この言葉が無限の大きさ比べのキーワードになります.

\begin{hamadadefi}[1対1対応]
$f$を$A$から$B$への写像とする.
任意の$b\in B$に対し$b=f(a)$を満たす$a\in A$がただ一つ存在するとき,
$f$を$A$から$B$への1対1対応
\footnote{一般には全単射(全射かつ単射の意)と呼ばれることが多いですが,わかりやすさを考えてこの表現を使いました.}
と呼ぶ.
\end{hamadadefi}

\begin{hamadaqst}
\label{bijecq}
例\ref{mapex}の写像$f$は$\mathbb{R}$から$\mathbb{R}$への1対1対応であることを示せ.
\end{hamadaqst}

\section{集合の大きさ比べ}
さて,いよいよ無限の大きさを比べてみましょう.
そこで,無限集合の「大きさ」にあたる濃度という概念を定義します.

\begin{hamadadefi}[濃度]
集合$A,B$に対し,$A$から$B$への1対1対応が存在するとき,
$A$と$B$の濃度は等しいと言い,このことを$|A|=|B|$と書く.
\end{hamadadefi}

ここで注意しておきたいのは,濃度は「一つひとつの集合に対して定まるものではない」
ということです.つまり,無限集合の大きさを,一つの集合の大きさを見るのではなく,
二つの集合のうちどちらが大きいかによって測ろうとしています.
「無限は数えられない」という問題をこのように回避するのは,
なかなかうまいやり方だと思いませんか.

\begin{hamadaex}
\label{NandZ}
$\mathbb{N}$は$\mathbb{Z}$の真部分集合であるが,
$|\mathbb{N}|=|\mathbb{Z}|$が成り立つ.
実際,写像$f\colon\mathbb{N}\to\mathbb{Z}$を
\[
f(n)=
\begin{cases}
-\frac{n}{2} & (nが偶数のとき) \\
\frac{n+1}{2} & (nが奇数のとき)
\end{cases}
\]
と定めれば,これは1対1対応である.$f$は次のような写像になっている.
\begin{table}[h]
\centering
\begin{tabular}{c||c|c|c|c|c|c|c|c}
$n$&0&1&2&3&4&5&6&$\cdots$ \\\hline
$f(n)$&0&1&$-1$&2&$-2$&3&$-3$&$\cdots$
\end{tabular}
\end{table}
\end{hamadaex}

\begin{hamadaqst}
\label{Nand2N}
$E=\{n\in\mathbb{N}\mid nは偶数\}$とする.
$E$は$\mathbb{N}$の真部分集合であることを示し,
さらに$|\mathbb{N}|=|E|$が成り立つことを示せ.
\end{hamadaqst}

例\ref{NandZ}や問題\ref{Nand2N}からわかるように,
$A\subsetneq B$であっても$|A|=|B|$となることがあります.
この一見不思議な現象は,有限集合では決して起こりません.
この事実だけでも,無限がいかに恐ろしいものかおわかりいただけるかと思います.

さて,ここまでは濃度の等しい集合ばかり見てきましたが,いよいよ大きさの違う無限の話題に入ります.

\begin{hamadadefi}[可算・非可算]
$|A|=|\mathbb{N}|$を満たす集合$A$を可算無限集合と呼ぶ.
このとき,$A$は可算である,といった言い方もする.
一方,可算でない無限集合のことを非可算無限集合と呼ぶ.
\end{hamadadefi}

例\ref{NandZ}で見た通り,$\mathbb{Z}$は可算となります.
実は$\mathbb{Q}$も可算であることが証明できます(考えてみてください).
しかし,$\mathbb{R}$は非可算です.最後にこのことを証明しましょう.

\begin{hamadalem}
\label{real01}
$|\mathbb{R}|=|(0,1)|$である.
ただし$(0,1)$は$\{x\in\mathbb{R}\mid0<x<1\}$を意味する.
\end{hamadalem}
\begin{proof}
写像
$f\colon(0,1)\to\mathbb{R}$, $x\mapsto\tan\left(x-\dfrac{1}{2}\right)\pi$
は1対1対応である.
したがって$|\mathbb{R}|=|(0,1)|$である.
\end{proof}

\begin{hamadathm}
\label{cantor}
$\mathbb{R}$は非可算である.
\end{hamadathm}
\begin{proof}
補題\ref{real01}より,$(0,1)$が非可算であることを示せばよい.
このことを背理法を用いて示そう.
$(0,1)$が可算であると仮定すると,$\mathbb{N}$から$(0,1)$への1対1対応が存在する.
すなわち,$(0,1)$の元を$x^{(0)},x^{(1)},x^{(2)},\cdots$と並べあげることができる.
以下$(0,1)=\{x^{(0)},x^{(1)},x^{(2)},\cdots\}$とする.

次に,各$n=0,1,2,\cdots$に対し$x^{(n)}\in(0,1)$を以下のように10進小数表示する:
\[
x^{(n)}=\sum_{k=0}^\infty x_k^{(n)}\cdot10^{-k-1}
=0.x_0^{(n)}x_1^{(n)}x_2^{(n)}x_3^{(n)}x_4^{(n)}x_5^{(n)}\cdots,\ 
x_k^{(n)}\in\{0,1,2,3,4,5,6,7,8,9\}
\]
ここで,$x^{(n)}\neq0,1$であることから,
$x_0^{(n)}=x_1^{(n)}=x_2^{(n)}=\cdots=0$や$x_0^{(n)}=x_1^{(n)}=x_2^{(n)}=\cdots=9$
が成り立つことはない.そこで
\[
y_n=
\begin{cases}
2 & (x_n^{(n)}が奇数のとき)\\
1 & (x_n^{(n)}が偶数のとき)
\end{cases}
(n=0,1,2,\cdots),\ 
y=\sum_{n=0}^\infty y_n\cdot10^{-n-1}
\]
とおくと,$y\in(0,1)$となる.ところが,$y$の作り方から,
すべての$n=0,1,2,\cdots$に対して$y\neq x^{(n)}$となることがわかる.
これは$(0,1)=\{x^{(0)},x^{(1)},x^{(2)},\cdots\}$であることに矛盾する.
したがって,$(0,1)$は非可算であるから,
濃度の等しい$\mathbb{R}$も非可算である.
\end{proof}

定理\ref{cantor}は,$\mathbb{R}$が非可算無限集合であることを示すための
有名な方法で,カントールの対角線論法と呼ばれるものの一種です.
この定理から,集合$\mathbb{R}$は$\mathbb{N}$や
$\mathbb{Z}$,$\mathbb{Q}$よりも「大きな」無限集合であることがわかります.

これで本文はおしまいです.お疲れ様でした.

\section{章末問題}
\begin{hamadaqst}[ド・モルガンの法則]
\label{demorgan}
$X$を集合とし,$A,B$をその部分集合とする.
このとき
\[ A^c=\{x\mid x\in X かつ x\notin A\} \]
により定義される集合$A^c$を$A$の補集合と呼ぶ.
また
\[ A\cup B=\{x\mid x\in A または x\in B\},A\cap B=\{x\mid x\in A かつ x\in B\} \]
により定義される集合$A\cup B$,$A\cap B$を,それぞれ$A$と$B$の和集合,共通部分と呼ぶ.
これらに関する次の等式を証明せよ.
\begin{enumerate}
\item $(A\cup B)^c=A^c\cap B^c$
\item $(A\cap B)^c=A^c\cup B^c$
\end{enumerate}
\end{hamadaqst}

\begin{hamadaqst}
\label{subsetequiv}
集合$X,Y$に対し,次の3つの条件は同値であることを証明せよ.
すなわち,次の3つの条件は互いに必要十分条件となっていることを示せ.
\begin{enumerate}
\item $X\subset Y$
\item $X\cup Y=Y$
\item $X\cap Y=X$
\end{enumerate}
\end{hamadaqst}

\begin{hamadaqst}[ラッセルのパラドックス]
\label{russel}
次のような集合を考えると矛盾が生じることを示せ.
\[ X=\{S\mid Sは集合でS\notin S\} \]
\end{hamadaqst}

\begin{hamadadefi}[二項関係]
$X$を集合とする.$x,y\in X$がある条件を満たすときに$x\sim y$
と書くとき,$\sim$を$X$の二項関係という.
\end{hamadadefi}

\begin{hamadadefi}[同値関係]
集合$X$に対し,その二項関係$\sim$が次の3つの条件を満たすとき,
$\sim$を$X$の同値関係と呼ぶ.
\begin{enumerate}
\item(反射律)任意の$x\in X$に対し$x\sim x$
\item(対称律)任意の$x,y\in X$に対し$x\sim y$ならば$y\sim x$
\item(推移律)任意の$x,y,z\in X$に対し$x\sim y$かつ$y\sim z$ならば$x\sim z$
\end{enumerate}
\end{hamadadefi}

\begin{hamadaqst}
\label{cardiseq}
次の問いに答えよ.
\begin{enumerate}
\item 「集合の大きさ比べ」の節において定義した「濃度」は反射律,対称律,推移律を満たすことを示せ.
すなわち,集合$A,B,C$に対し以下が成り立つことを示せ.
\begin{enumerate}
\item $|A|=|A|$
\item $|A|=|B|$ならば$|B|=|A|$
\item $|A|=|B|$かつ$|B|=|C|$ならば$|A|=|C|$
\end{enumerate}
\item 「集合の大きさ比べ」の節において,1の事実を利用している箇所を指摘せよ.
\end{enumerate}
\end{hamadaqst}

\begin{hamadaqst}
\label{cantorpractice}
実数列全体の集合を$\mathbb{R}^\mathbb{N}$とするとき,$\mathbb{R}^\mathbb{N}$は非可算であることを証明せよ.
\end{hamadaqst}

\section{問題の解答}
\Subsection{問題\ref{shugoq}}
(例)
$X=\{x\mid xは正三角形\}$は集合である.
一方,$Y=\{y\mid yは円に近い図形\}$は集合とは言えない.

\Subsection{問題\ref{mapq}}
(例)
写像$f\colon\mathbb{N}\to X$を
$f(n)=(nを4で割ったときの余り)$と定めると,$f$は写像である.
一方,$n\in\mathbb{N}$に対し$g(n)=(n個の元を持つXの部分集合)$とすると,$g$は写像でない.

\Subsection{問題\ref{bijecq}}
任意の$y\in\mathbb{R}$に対し,$y=f(x)$すなわち$y=2x+1$を満たす
$x\in\mathbb{R}$は,$x=\dfrac{y-1}{2}$ただ一つである.
したがって,$f$は$\mathbb{R}$から$\mathbb{R}$への1対1対応である.

\Subsection{問題\ref{Nand2N}}
まず$E\subsetneq\mathbb{N}$であることを示そう.
$E\subset\mathbb{N}$であることは$E$の定義からわかるから,
$E\neq\mathbb{N}$であること,すなわち$\mathbb{N}\subset E$が
成り立たないことを示せばよい.
これは次のようにしてわかる:
1は$\mathbb{N}$の元であるが$E$の元ではない.
したがって$\mathbb{N}\subset E$は成り立たない.
ゆえに$E\subsetneq\mathbb{N}$である.

次に$|\mathbb{N}|=|E|$であることを示そう.
写像$f\colon\mathbb{N}\to E$を
\[
f(n)=2n\ (n=0,1,2,\cdots)
\]
と定めると,これは$\mathbb{N}$から$E$への1対1対応である.
したがって$|\mathbb{N}|=|E|$である.

\section{章末問題の解答}
\Subsection{問題\ref{demorgan}}
\begin{enumerate}
\item $x\in(A\cup B)^c$に対し,$x\notin A\cup B$なのだから$x\notin A$かつ$x\notin B$が成り立つ.
したがって$x\in A^c\cap B^c$である.
また,$x\in A^c\cap B^c$に対し,$x\notin A$かつ$x\notin B$なのだから$x\notin A\cup B$が成り立つ.
したがって$x\in(A\cup B)^c$である.
以上より$(A\cup B)^c=A^c\cap B^c$である.
\item $x\in(A\cap B)^c$に対し,$x\notin A\cap B$なのだから$x\notin A$または$x\notin B$が成り立つ.
したがって$x\in A^c\cup B^c$である.
また,$x\in A^c\cup B^c$に対し,$x\notin A$または$x\notin B$なのだから$x\notin A\cap B$が成り立つ.
したがって$x\in(A\cap B)^c$である.
以上より$(A\cap B)^c=A^c\cup B^c$である.
\end{enumerate}

\Subsection{問題\ref{subsetequiv}}
まず$1\Rightarrow2$であることを示そう.
$Y\subset X\cup Y$は明らかだから,$X\cup Y\subset Y$であることを示せば十分である.
そこで,$x\in X\cup Y$とすると$x\in X$または$x\in Y$である.
$x\in X$であるとすると,仮定1より$X\subset Y$だから$x\in Y$である.
したがって,いずれにしても$x\in Y$である.
ゆえに$X\cup Y\subset Y$である.
よって2が成り立つ.

次に$2\Rightarrow3$であることを示そう.
$X\cap Y\subset X$は明らかだから,$X\subset X\cap Y$であることを示せば十分である.
そこで$x\in X$であるとすると特に$x\in X\cup Y$である.
よって,仮定2より$X\cup Y=Y$だから$x\in Y$である.
したがって$x\in X\cap Y$であり,ゆえに$X\subset X\cap Y$である.
以上より3が成り立つ.

最後に$3\Rightarrow1$であることを示そう.
そこで$x\in X$であるとすると,3より$x\in X\cap Y$である.
よって特に$x\in Y$である.
したがって1が成り立つ.

これで1,2,3が互いに必要十分条件になっていることが証明された.
実際,例えば$1\Leftrightarrow2$は次のようにしてわかる.
\begin{itemize}
\item $1\Rightarrow2$は直接示してある.
\item $2\Rightarrow1$は$2\Rightarrow3$かつ$3\Rightarrow1$であることから従う.
\end{itemize}

\Subsection{問題\ref{russel}}
もし$X\in X$であるとすると,$X$の定義より$X\notin X$となり矛盾する.
また,$X\notin X$であるとしても,やはり$X$の定義より$X\in X$となり矛盾する.

\Subsection{問題\ref{cardiseq}}
\begin{enumerate}
\item
\begin{enumerate}
\item $a\in A$を$a$自身に移す写像$\mathrm{id}\colon A\to A$を考えると
\footnote{このような写像のことを恒等写像と呼びます.},
これは1対1対応である.
したがって$|A|=|A|$である.
\item 仮定より,1対1対応$f\colon A\to B$が存在する.
このとき,すべての$b\in B$に対し$b=f(a)$を満たす$a\in A$がただ1つ存在するから,
$b\in B$に対し$b=f(a)$なる$a\in A$を対応させる写像$g\colon B\to A$を考えれば
\footnote{このような写像を逆写像と呼び,
$f$の逆写像のことをよく$f^{-1}$と書きます.この解答では$g=f^{-1}$となっています.},
$g$は1対1対応である.
したがって$|B|=|A|$である.
\item 仮定より,1対1対応$f\colon A\to B$および$g\colon B\to C$が存在する.
ここで,写像$h\colon A\to C$を$h(a)=g(f(a))$で定義すると
\footnote{このような写像を合成写像と呼び,よく$g(f(a))=(g\circ f)(a)$と書きます.},
これは1対1対応である.
したがって$|A|=|C|$である.
\end{enumerate}
\item 定理\ref{cantor}の証明の1文目で使っている.
もし$(0,1)$が可算すなわち$|(0,1)|=|\mathbb{N}|$であるとすると,補題\ref{real01}の結果から,
推移律を用いて$|\mathbb{R}|=|\mathbb{N}|$が得られ,$\mathbb{R}$は可算となってしまう.
\end{enumerate}

\Subsection{問題\ref{cantorpractice}}
背理法により証明する.
$\mathbb{R}^\mathbb{N}$が可算であると仮定すると
\[ \mathbb{R}^\mathbb{N}=\{x^{(0)},x^{(1)},x^{(2)},\cdots\} \]
と書ける.
ただし,各$m\in\mathbb{N}$に対し$x^{(m)}=\{x_n^{(m)}\}_{n=0}^\infty$は実数列である.
このとき,数列$\{y_n\}_{n=0}^\infty$を次のように定める.
\[ n=0,1,2,\cdots に対し y_n=
\begin{cases}
1 & (x_n^{(n)}が負であるとき) \\
-1 & (x_n^{(n)}が負でないとき)
\end{cases} \]
$\{y_n\}_{n=0}^\infty$は実数列であるから$\{y_n\}_{n=0}^\infty\in\mathbb{R}^\mathbb{N}$である.
ところが,$\{y_n\}_{n=0}^\infty$の定義から,各$m\in\mathbb{N}$に対し$x^{(m)}\neq\{y_n\}_{n=0}^\infty$となる.
したがって$\{y_n\}_{n=0}^\infty\notin\mathbb{R}^\mathbb{N}$であるが,これは矛盾である.
以上より,$\mathbb{R}^\mathbb{N}$は非可算である
\footnote{対角線論法を用いるひとつの例として出題しました.}.

\end{document}

%\begin{thebibliography}{9}
%\item Hull, J. C. (2014), Options, Futures, and Other Derivatives, 9th edition (Upper Saddle River, NJ: Prentice Hall).
%\end{thebibliography}
