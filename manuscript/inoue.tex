\documentclass{jreport}

\usepackage{amsmath,amsthm}
\usepackage{amssymb}
\usepackage{multicol}
\usepackage[dvipdfmx]{graphicx}
\usepackage[top=25truemm,bottom=20truemm,left=20truemm,right=20truemm]{geometry}


\theoremstyle{idefinition} 
\newtheorem{idefi}{定義}[section] 
\newtheorem{iqst}[idefi]{問題} 
\newtheorem{iprop}[idefi]{命題} 
\newtheorem{ithm}[idefi]{定理}  
\renewcommand\proofname{\bf 証明} 

 



\begin{document}

\chapter{Clifford代数(井上)}

\section{前書き}
\ こんにちは,数学科の井上と申します.e$\pi$isodeに書くテーマを決めたときは代数を勉強するつもりでテーマを選んだのですが,残念ながら今はあまり代数に触れていません.自分で書いておきながら,これが実際になんの役に立つのかと聞かれても私もよくわかってないという状態です.というわけで発展的な内容までは書けませんがお付き合いいただければ幸いです.\\

\section{数}
\ 数にはいろんな種類があります.例えば以下のような数の集合が有ります.(集合の概念,表し方等は濱田さんのページのあるのでそちらを参照してください.)\\
\ 自然数 $\mathbb{N} = \{(0),1,2,3,\cdots \}$\\
\ 整数 $\mathbb{Z} = \{\cdots-1,0,1,2,3,\cdots \}$\\
\ 有理数 $\mathbb{Q} = \{\frac{p}{q}\ |\ p\in \mathbb{N},q\in\mathbb{N}/\{ 0\} \}$\\
\ 実数 $\mathbb{R}=\{1,2,\sqrt{2},\pi,\cdots \}$\\
\ さらに実数だけではすべての有理数係数の代数方程式($a_nx^n+\cdots +a_1x+a_0=0$の形をした方程式)は解けないので,複素数という考え方が導入されます.二乗すると$-1$になる$i$という記号を用います.(例えば$x^2+1=0$の解は$\pm i$になのでこれは実数解は持たないが複素数解をもつ.)\\
\ 複素数 $\mathbb{C}=\{i ,2+3i ,\cdots \}$\\
\ 今回はさらにこの''数''を広げていくことを考えます.\\

\begin{idefi} 
\ 集合$F$が以下の条件を満たすとき体と呼ぶ.ただし$F$に入る演算を加法$+$と乗法$\times$とし,a,b,cを任意の$F$の元とする.\\
\ 1. 加法の結合法則$a+(b+c)=(a+b)+c$を満たす\\
\ 2. $a+0=0+a=a$となる零元$0$が存在する\\
\ 3. $a+(-a)=(-a)+a=0$となる加法の逆元$(-a)$が存在する\\
\ 4. 加法の交換法則$a+b=b+a$を満たす\\
\ 5. 乗法の結合法則$a\times(b\times c)=(a\times b)\times c$を満たす\\
\ 6. $a\times1=1\times a$となる単位元$1$が存在する\\
\ 7. $a\times a^{-1}=a^{-1} \times a=1$となる乗法の逆元$a^{-1}$が存在する(ただし$a\neq 0$)\\
\ 8. 乗法の交換法則$a\times b=b\times a$を満たす\\
\ 9. 分配法則$a\times(b+c)=a\times b+a \times b$
\end{idefi} 

\ この定義に基づくと整数は整数の中に乗法の逆元を持たない数があるので体ではありません.(例えば$2$の乗法の逆元は$\frac{1}{2}$だがこれは整数の集合に含まれない.)一方有理数,実数,複素数はすべての条件を満たしているので体になります.体は引き算や割り算が無条件にできるなど性質が良いので扱いやすです.ちなみに複素数$\mathbb{C}$は任意の複素数係数の代数方程式の解は複素数に含まれるということから,代数的に閉じているので代数閉体と呼ばれます.\\



\section{Galois拡大}
\ それでは,この複素数$\mathbb{C}$をさらに拡張することはできるのでしょうか.まずは$\mathbb{C}$を含むような体が存在しないか考えてみます.体を拡大する方法はいくつかありますがGalois拡大というものに焦点を当ててみたいと思います.\\

\begin{idefi}
\ Galois拡大\\
\ E,Fを体とする.このとき拡大E/Fが代数拡大であって,正規拡大かつ分離拡大であるもの.\\
\ 代数拡大とはEのすべての元がF係数の代数方程式の解になっていること.正規拡大とは,Eに根を持つF係数の既約な多項式はすべてE上で一次式の積に分解できる.これはある元がEに含まれていたらそれと共役な元も常にEに含まれるということ.分離拡大とは任意の$a \in E$の最小多項式(aを根にもつ多項式のうち最も次数の小さいもの)がE上で相異なる一次式の積に分解できること.
\end{idefi}

\ 例えば$E=\mathbb{C},F=\mathbb{R}$とおくと確かに$\mathbb{C}/\mathbb{R}$はGalois拡大になっています.代数拡大であることについては,$a+bi\in \mathbb{C}$は$x^2+2ax+a^2-b^2=0$の解になっていることなどからいえます.正規拡大であることも$a+bi\in \mathbb{C}$なら$a-bi\in \mathbb{C}$となるのでいえます.そして分離拡大になっていることも$a+bi\in \mathbb{C}$の最小多項式は$x^2+2ax+a^2-b^2=(x-a+bi)(x-a-bi)$となっていることなどにより言えます.\\


\begin{idefi}
\ 拡大次数\\
\ F上のベクトル空間としてのEの次元を拡大次数という.\\
\end{idefi}



\begin{iprop}
\ $\mathbb{R}$のGalois拡大は2次拡大の$\mathbb{C}$(とその同型)のみである.\\
\end{iprop}


\begin{proof}
\ まず2次拡大であることはイメージだけ書きます.$\mathbb{C}$の元は全て$\mathbb{R}$の元a,bを用いて$a+\sqrt{-1} b$と書けるので,$\mathbb{C} \simeq \mathbb{R} \oplus \mathbb{R} \sqrt{-1}$ つまり$\mathbb{R}$が二つくっついたものと見なせるということ.\\
\ ここから$\mathbb{R}$のGalois拡大は$\mathbb{C}$のみになることを順に示します.最初に$\mathbb{R}$の3次以上の奇数次Galois拡大は存在しないことを示す.f(x)を$\mathbb{R}$係数の3次以上の奇数次元多項式とする.($f(x)=a_nx^n+\cdots +a_1x+a_0$としたときに$n$は3以上の奇数.)このとき$\displaystyle \lim_{x \to +\infty} f(x)=+\infty$ $\displaystyle \lim_{x \to -\infty} f(x)=-\infty$なので,中間値の定理より$f(a)=0$となる$a\in \mathbb{R}$が存在し,$f(x)=(x-a)g(x)$と因数分解されるので既約ではない.よって$\mathbb{R}$の3次以上の奇数次の既約多項式は存在しない.\\
\ 次に$\mathbb{C}$の2次拡大は存在しないことを示す.もし$\mathbb{C}$が2次拡大を持つなら,$f(x)=x^2+ax+b$という形の既約多項式がある.しかし,$f(x)=(x-\frac{-a+\sqrt{a^2-4b}}{2})(x-\frac{-a-\sqrt{a^2-4b}}{2})$と$\mathbb{C}$上で因数分解できてしまう.よって矛盾. $\mathbb{C}$は$\mathbb{R}$の拡大体なので,もし$\mathbb{C}$の拡大体があれば$\mathbb{R}$の拡大体にもなっている.そのため$\mathbb{R}$の偶数次の拡大体は存在しない.あとは$\mathbb{R}$の次拡大は$\mathbb{C}$(とその同型)のみということを示せばこの証明は終わりますが今回は略します.\\
\end{proof}

\ ちなみにこのipropは任意の$\mathbb{R}$係数の多項式が$\mathbb{C}$上で一次式の積に分解できることも意味している.よって,任意のn次の$\mathbb{R}$係数の代数方程式が$\mathbb{C}$上にn個の解を持つという代数学の基本定理がこれによっても示される.(代数学の基本定理は多くの異なる証明が与えられていることでも有名.)\\


\newpage

\section{Clifford代数}

\ それではGalois拡大はあきらめて別の方面から考えてみます.\\
\ 知っている人も多いと思いますがHamiltonの四元数$\mathbb{H}$を考えます.これは二乗すると$-1$になる数,i,j,kを用いて$a+bi+cj+dk$(a,b,c,dは実数)の形で書かれます.このときi,j,kの間の掛け算は次のようになります.\\

\begin{center}
\begin{tabular}{|c|c|c|c|c|}\hline
(左側)$\times$(右側) & 1 & i & j & k \\\hline
1 & 1 & i & j & k \\\hline
i & i & -1 & k & -j \\\hline
j & j & -k & -1 & i \\\hline
k & k & j & -i & -1 \\\hline
\end{tabular}
\end{center}


\ この表を見てわかるように$i\times j=k$だが$j\times i=-k$なので,掛け算は交換できない(非可換)になっています.よって体の8個目の条件である乗法の交換法則を満たしていません.このようなものは斜体と呼ばれます.$\mathbb{H}$は$\mathbb{C}$を拡大したものだが$\mathbb{H}$は体ではないということです.\\ 

\ それでは,体ではなく斜体だったらどこまでも拡張できるのでしょうか.答えはyesです.Clifford代数($Cln$)で拡張できます.これによりいくつかの数を一般化することができます.\\

\begin{idefi}
\ Clifford代数\\
\ $Cln$を$\mathbb{R}-$線形空間とする.\\
\ $Cln$の基底のうちのn個を$e_1,e_2,\cdots ,e_n$としたとき積を以下を線形に拡張したもので定義する.\\
\begin{eqnarray}
\left\{
\begin{array}{l}
e_s e_r=-e_s e_r (s\neq r , 1\leq r,s \leq n)\\
e_s^2=-1
\end{array}
\right.
\end{eqnarray}
\ このとき$Cln$の基底は$\{1,e_1,e_2,\cdots,e_n,e_1e_2,e_2e_3,\cdots,e_{n-1}e_n,\cdots,e_1e_2\cdots e_n \}$の$2^n$個になります.よって次元は基底の個数である$\dim{Cln}=2^n$になります.\\
\end{idefi}

\ n=0の時は$Cl_0=\mathbb{R}$となります.\\
\ n=1の時は$Cl_1=\mathbb{C}$となります.このとき$e_1=i$になっています.\\
\ n=2の時は$Cl_2=\mathbb{R}$となります.このとき$e_1=i,e_2=j,e_1e_2=k$となっています.\\
\ このようにして$n=3,4,5,\cdots$ と定義していくことができます.\\

\ 上で定めた基底で$Cln$を$2^n$次実正方行列と同一視することができます.\\
\ $n=0$つまり$\mathbb{R}$の時はそのまま実数を一次元実正方行列と見なします.\\
\ $n=1$つまり複素数体$\mathbb{C}$のときは以下の対応がつけられます.\\

\ $1\leftrightarrow $
$ \begin{pmatrix}
 1 & 0 \\
 0 & 1 \\
\end{pmatrix}  $ ,
\ $i \leftrightarrow $
$ \begin{pmatrix}
 0 & -1 \\
 1 & 0 \\
\end{pmatrix}  $\\
\ つまり
\ $a+bi\leftrightarrow $
$ \begin{pmatrix}
 a & -b \\
 b & a \\
\end{pmatrix}  $
\ となります.この対応は一対一になります.\\

\ $n=2$つまり$\mathbb{H}$の時も以下のように一対一対応になります.\\

\ $1\leftrightarrow $
$ \begin{pmatrix}
 1 & 0 & 0 & 0 \\
 0 & 1 & 0 & 0 \\
 0 & 0 & 1 & 0 \\
 0 & 0 & 0 & 1 \\
\end{pmatrix}  $ ,
\ $i \leftrightarrow $
$ \begin{pmatrix}
 0 & -1 & 0 & 0 \\
 1 & 0 & 0 & 0 \\
 0 & 0 & 0 & -1 \\
 0 & 0 & 1 & 0 \\
\end{pmatrix}  $ ,
\ $j \leftrightarrow $
$ \begin{pmatrix}
 0 & 0 & -1 & 0 \\
 0 & 0 & 0 & 1 \\
 1 & 0 & 0 & 0 \\
 0 & -1 & 0 & 0 \\
\end{pmatrix}  $ ,
\ $k \leftrightarrow $
$ \begin{pmatrix}
 0 & 0 & 0 & -1 \\
 0 & 0 & -1 & 0 \\
 0 & 1 & 0 & 0 \\
 1 & 0 & 0 & 0 \\
\end{pmatrix}  $ \\\\\\\\





\ おまけ\\
\ 証明等はしませんが$n=8$までの$Cln$が何と同型になっているかだけあげておきます.ただし$M_n(K)$はKを行列成分に持つn次正方行列とします.\\
\begin{center}
\begin{tabular}{|c|c|c|}\hline
$Cln$ & 同型なもの & 次元 \\\hline
$Cl_0$ & $\mathbb{R}$ & 1 \\\hline
$Cl_1$ & $\mathbb{C}$ & 2 \\\hline
$Cl_2$ & $\mathbb{H}$ & 4 \\\hline
$Cl_3$ & $\mathbb{H}\times \mathbb{H}$ & 8 \\\hline
$Cl_4$ & $M_2(\mathbb{H})$ & 16 \\\hline
$Cl_5$ & $M_4(\mathbb{C})$ & 32 \\\hline
$Cl_6$ & $M_8(\mathbb{R})$ & 64 \\\hline
$Cl_7$ & $M_8(\mathbb{R}) \times M_8(\mathbb{R})$ & 132 \\\hline
$Cl_8$ & $M_16(\mathbb{R})$ & 256 \\\hline
\end{tabular}
\end{center}


\section{参考文献}
\ Matrix Groups: An Introduction to Lie Group Theory / Andrew Baker (2002) \\
\ 『代数学〈1〉群と環 (大学数学の入門)』  桂 利行 (2004)







\end{document}